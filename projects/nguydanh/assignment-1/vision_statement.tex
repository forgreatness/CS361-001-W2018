\documentclass[12pt]{article}

%% Language and font encodings
\usepackage[english]{babel}


%% Sets page size and margins
\usepackage[a4paper,top=3cm,bottom=2cm,left=3cm,right=3cm,marginparwidth=1.75cm]{geometry}

%% Useful packages
\usepackage{graphicx}
\usepackage[colorlinks=true, allcolors=blue]{hyperref}
\usepackage{setspace}   %Allows double spacing with the \doublespacing command
\usepackage{indentfirst}

\title{Askie Forum}
\author{
         name: Danh Nguyen
         name: Joshua Bell 
         \linebreak {onid: nguydanh}
         \linebreak {onid: belljos}
    }

\begin{document}
\maketitle

\tableofcontents
\section{Vision Statement}
\doublespacing
Have you ever felt annoyed or angry at people who seem to ask the most ridiculous questions in class? I have certainly felt annoyed at some of these people, but I at the same time have also been a part of these group of people. Why is it that we feel annoy when others ask questions? Doesn.t questions promote learning, and help clarified confusion. Although these questions may help other students thrive in their learning, it may however discourage those whom are disrupted or distracted by these questions. These distraction leads to complaints that may frighten and intimidate the crowd that needs to ask questions because they feel like they might be holding everyone else back.

According to an article by Tenney School call .When Students Do Not Ask Questions in Class. these intimidation leads to problems and issues that people face when they want to ask question such as .Shyness, fear of peers, fear of appearing dumb, difficulty forming the question, consequences, and hinders self-esteem. (When). Clearly as described above, we can see that these issues arise from two sides. One is the people whom seem to need help and the other is everybody else. Because of this I want a solution that will solve the issues the first side is facing, while also pleasing the rest. The proposed solution is called an Askie Forum, which is a medium where ideas can be exchange. Within this forum, the user can anonymously request an answer from the poll of users whom are within that certain medium. This certain medium can be classified as a channel just like clickers. Being anonymous will encourage more people to ask questions because Askie Forum help them eliminate some of the problems and issues they face when they want to ask a question. However, we also need to find a way to encourage the rest of users within the medium to participate in the forum and help the other users clear up their confusion. Some of ways we can do this are by requiring all users within the setting to participate in the Forum to receive credits for their class or be allow within the setting. At the end of the settings, the event creator will have a chance to look at some of the questions that may have not been answer by everyone.  Doing this will help reveal some questions that everyone might need an answer to, while also eliminate those questions where it might be non-relevant to the topics of the settings. What makes Askie Forum different from all other similar forum is the ability for the user to acquire a solution at the end of a certain time frame. To do this, we can add a feature to the question to allow the user to mark their question as urgent or non-important. By marking a question as urgent, if the questions which are urgent cannot be answer by peers within a certain time set by user, the event creator will be alert and choose to answer the question if it is relevant and whenever convenient. Other questions that are not answered will be responded by event creator at the end of the event.

Certainly, there are other similar solution. But what may completely help separate Askie Forum from the rest of its competitor, is the ability to acquire an answer fast and accurate. This ability the user acquire from Askie Forum is granted by the purpose of using Askie Forum. Which its purpose is to be used to help the event coordinator clear up confusion and help promote a better event. 

\section{Reference}
\begin{flushleft}
[1] ..When Students Do Not Ask Questions in Class.. Tenney School, Tenney School, 18 May 2017.
\url{http://tenneyschool.com/when-students-do-not-ask-questions-in-class/}
\newline
\end{flushleft}

\end{document}
